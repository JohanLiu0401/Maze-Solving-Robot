% 前所未有简单地,开始你的 LaTeX 之旅

%%%%%%%%%%%%%%%%%%%%%%%%%%%%%%%%%%%%%%%%%%%%%%%%%%%%%%%%%%%%%%%%%%%%%%%%%
% [页面选项]
%
% 字号选项:    11pt是字号,比较适合电子印刷品和纸质印刷品
% 字体选项:    不填写 - 默认为 Modern (别忘了连逗号一起删了)
%			  times - Times New Roman
%			  sans - Sans Serif 字体(黑体)
% oneside:    代表单面,一般单侧用于电子印刷品
% 		      可以改成 twoside,一般用于双面纸质印刷
% openright:  双面打印的情况下,纸质印刷品要求新的一章从奇数页开始
% hardcopy:   打印机打印选项,带有此选项将去除logo颜色以及代码颜色,更适合黑白印刷
%			  使用彩色或者电子版请去除此选项
%			  也可以改为 editing ,颜色讲和 sublime3 的 Mariana 颜色主题一致,减小写作时的文本对比度
%		          让你以最舒适的方式书写
% heading:    页面带有页眉
\documentclass[11pt,times,oneside,openright,hardcopy]{eeereport}
%%%%%%%%%%%%%%%%%%%%%%%%%%%%%%%%%%%%%%%%%%%%%%%%%%%%%%%%%%%%%%%%%%%%%%%%%


%%%%%%%%%%%%%%%%%%%%%%%%%%%%%%%%%%%%%%%%%%%%%%%%%%%%%%%%%%%%%%%%%%%%%%%%%
% [封面选项]
%
% title:    文章主标题,不可以省略
% subtitle: 副标题,可以省略,但不可以删除
% covercontent: 封面信息,可以在其内部添加或者删除 \converline 来添加或者删除一行
%    coverline{项目}{项目内容},可以自由添加各种内容
%
\title{Specification and Design Report}
\subtitle{}
\covercontent{
	\coverline{Author}{Zhiyong Liu}
	% \coverline{Partner}{Your Partner}
	\coverline{Module}{CSE305}
	\coverline{Supervisor}{Dr. Jieming Ma}
	\coverline{Date}{$25^{th}$~/~Nov~/~2018}
}
%%%%%%%%%%%%%%%%%%%%%%%%%%%%%%%%%%%%%%%%%%%%%%%%%%%%%%%%%%%%%%%%%%%%%%%%%

%%%%%%%%%%%%%%%%%%%%%%%%%%%%%%%%%%%%%%%%%%%%%%%%%%%%%%%%%%%%%%%%%%%%%%%%%
% 页面设定
% 如果在前面设定了需要页面,请去掉下面三个的注释,然后分别进行定义,如果不定义,
% LaTeX 会自动使用章节名称来进行页面定义
%\lhead{} 
%\chead{} 
%\rhead{}
%%%%%%%%%%%%%%%%%%%%%%%%%%%%%%%%%%%%%%%%%%%%%%%%%%%%%%%%%%%%%%%%%%%%%%%%%


\begin{document}
% set line spacing to 1.5B
\baselineskip = 17pt
\pagenumbering{Alph}
\begin{titlepage}
% 不要编译!该文件已经包含在 report.tex 里了
\thispagestyle{empty}
\newcommand\nbvspace[1][3]{\vspace*{\stretch{#1}}}
\newcommand\nbstretchyspace{\spaceskip0.5em plus 0.25em minus 0.25em}
\newcommand{\nbtitlestretch}{\spaceskip0.6em}
\newcommand{\psubtitle}[1]{\LARGE\textbf{{#1}}\\\normalsize}
\newcommand{\ptitle}[1]{\huge\textbf{{#1}}\\\normalsize}

\begin{center}
\nbvspace[0.5]
\begin{figure}[h]
\centering
\includegraphics[width=10cm]{\logofile}
\label{fig:logo}
\end{figure}
%\psubtitle{Experimental Report}
\nbvspace[1]
\ptitle{\rtt}
\nbvspace[0.5]
\psubtitle{\srtt}
\nbvspace[1]

\end{center}
\begin{flushleft}

\begin{table}[!hbp]
\centering
\begin{tabular}{p{2.5cm} p{7cm}}   \\
  \cct
 \end{tabular}
\end{table}
\normalsize
\nbvspace[0.5]
\end{flushleft}

\newpage
\thispagestyle{empty}
\end{titlepage}

\frontmatter

% \tableofcontents
% \addcontentsline{toc}{chapter}{Contents}
	
%%%%%%%%%%%%%%%%%%%%%%%%%%%%%%%%%%%%%%%%%%%%%
% 摘要部分 Abstract
%%%%%%%%%%%%%%%%%%%%%%%%%%%%%%%%%%%%%%%%%%%%%
% \chapter{Abstract}

% Lorem ipsum dolor sit amet, consectetur adipiscing elit. Cras congue massa a ex luctus, in consectetur velit venenatis. Praesent porta dolor augue, at eleifend purus accumsan a. Vestibulum ullamcorper massa eget lobortis volutpat. Vestibulum at luctus lectus, vitae tristique ligula. Pellentesque ultrices viverra tellus, ut hendrerit urna varius ac. Sed nulla tortor, dignissim in justo nec, ullamcorper ultricies magna. In finibus, nisi eu lobortis ultricies, neque elit tempus purus, sit amet condimentum lorem ligula id magna. Integer a magna at augue euismod rhoncus mattis sit amet velit. Morbi eget leo a dolor porttitor commodo non non velit. Lorem ipsum dolor sit amet, consectetur adipiscing elit. Curabitur sed quam feugiat, sollicitudin sapien id, semper est. In consectetur est eu sapien scelerisque, et sagittis sapien finibus. Nullam facilisis molestie ligula sed pharetra. Fusce nec sagittis arcu. Sed eu leo vel metus volutpat tincidunt. Suspendisse vehicula in elit eu pretium.

% Nulla sagittis urna venenatis ex lacinia molestie. Etiam tempus enim tortor, sed sollicitudin nisi tempus sit amet. Cras at metus massa. Sed consequat risus metus, ut placerat nisi gravida a. Donec quis justo eget risus maximus malesuada. Donec id urna sed eros porttitor ornare vitae non dolor. Donec mollis, leo sit amet aliquet posuere, velit lacus luctus ante, vel hendrerit eros mauris a ligula. Donec tincidunt dolor non enim malesuada, in molestie enim ornare. Aliquam gravida orci mauris, non scelerisque ante finibus in. Curabitur dictum dapibus mauris nec volutpat. Curabitur congue urna faucibus, feugiat lorem et, faucibus ante. Sed convallis erat ac nunc rhoncus, at fermentum velit efficitur. Nullam dolor metus, feugiat id consectetur vel, sagittis non nunc. Aliquam rutrum placerat ante eget blandit.

% \textbf{Key Words:} \LaTeX{}, consectetur, rutrum.

\mainmatter
\pagenumbering{arabic}

%%%%%%%%%%%%%%%%%%%%%%%%%%%%%%%%%%%%%%%%%%%%%
% 正文部分 Main Content
%%%%%%%%%%%%%%%%%%%%%%%%%%%%%%%%%%%%%%%%%%%%%
\chapter{Specification}\label{cpt:spec}

\section{Project Description}
Now more and more people are supposed to learn knowledge about programming. However, learning programming can sometimes become boring that causes learners give up halfway.
This project develops a maze game for those people who wish to learn programming knowledge. Users can learn programming knowledge by playing games at the same time.

In this project, there is a robot which is based on Raspberry Pi. This robot is to be an explorer in the maze and needs to find a path from the start position to the end position.
To run out of the maze, robot is supposed to have an algorithm to execute. This project aims to design a maze solving algorithm and achieves the algorithm on the robot.
In addition to this, the code programmed in this project also allows the users to design their own algorithm and execute it on robot. Users modify the codes according to algorithm designed and test it in the real maze, which achieves the aim of programming education.

Many algorithms such as wall follower, Pledge algorithm \cite{Klein:2011hi}, and Trémaux's algorithm \cite{Anonymous:2007ch}, were invented specially to deal with the maze solving problem, and each of them have their own strengths and weaknesses.
Besides, a maze can be viewed as a tree or graph, some algorithms used in graph theory also have the ability to solve the maze solving algorithm. One of them is Depth-first search algorithm, it is used to traverse the tree or graph data structure. Therefore, through the Depth-first algorithm, the maze can be traversed by the robot and the path from origin to destination eventually can be found.

This project are supposed to develop a maze solving algorithm based on Depth-first search algorithm and work accurately on the robot in the real maze.

\section{Statement of Deliverables}
% \subsection{Anticipated Documentation}
The Deliverable upon completion of the project is a software. The software is written in python and has multiple functionalities. In the first place, it is responsible for guiding the robot out of the maze.
The software is intended to be configured in the robot in advance. When the robot moves in the maze, software will give the next step command for robot to execute according to maze solving algorithm.
Furthermore, the software allows users to modify pre-configured algorithm and design their own maze solving algorithm. The pre-configured algorithm refers to the algorithm based on the Depth-first search algorithm designed in this project.
After users have learned the pre-configured algorithm, they can improve the default algorithm and run the new algorithm in the robot.

To evaluate the project, the first thing to do is to test if the designed algorithm is able to work accurately in maze solving problem. More specifically, the robot pre-configured with the software will be tested in multiple different mazes.
In all tests, the number of getting out of the maze will be recorded. By calculating the success probability, the project will be measured if it can solve the maze problem.
Besides, since this is a game-based learning project, the feedbacks of users will be collected. The feedbacks contain multiple aspects:
\lists{number}{
	\item Whether users think the game is interesting.
	\item Whether they think they can learn programming effectively through this game.
	\item Whether they are satisfied with the project.
	\item What other aspects they think for this project to improve.
}


\section{Conduct of Project and Plan}
\subsection{Background}
As computer and mobile technologies advance, a amount of games for educational purposes have been used among learners of different levels \cite{Proulx:2018fr}. 
This form of learning is called game-based learning (GBL). It has become the best solution for soft skills learning when traditional learning are homiletic, expensive and difficult to implement.
Game-based learning aims to balance subjects with gameplay and players' ability to retain and apply subjects to the real world \cite{Ifenthaler:2012tn}. 
Children prefers to cost much time on playing, and learns steps of digital games. In this way, they can play and learn at the same time.

Games have been used in education for a long time. In the middle ages, noblemen learned strategies of war by using ancient chess game. 
During the Civil War, volunteers from Rhode Island played American Kriegsspiel, which had originally been created in 1812 for training Prussian officers-of-war.
In order to train for Prussian officers-of-war, American Kriegsspiel was invented in 1812 \cite{Fenn:2014dm}.
Until now, a wide range of game-based learning applications are used, such as exploring ancient history with video games,
teaching empathy with video games \cite{Prensky:2007wt}.

The development of sophisticated digital gameing technologies in recent years has generated an \$8.1B industry around for educational purposes \cite{An:2018uv}. 
A lot of research on how game-based learning can enhance teaching are carried out. For instance, in order to encourage pupils to read frequently and
improve their reading skills effectively, the reasearchers contains a game-like exercise in a prototype using Multimedia Fusion Developer 2.
The game-like exercise is that foam volcano character spewes bubbles includes letters and words \cite{AdrirScott:2013ui}. Then, children must read them loudly to open up.

% \subsection{Design method} 后面写

\subsection{Implementation}
This project contains three major components, which are maze robot, algorithm, and maze environment.
The maze robot is built based on raspberry pi, which can be controlled by software. The maze robot is equipped with multiple types of sensors, such as optical sensors, range sensors and so on. With the sensors, maze robot has the ability to achieve wall detection. The robot manufacturer provides the API of robot action, which can be used to realize actions of robot, such as moving, making a turn.
The algorithm refers to the maze solving algorithm, which has been mentioned in project description. This algorithm is based on Depth-first search and considers the details in actual maze. The algorithm can be modified and improved by the users. Moreover, users are able to design their own maze solving algorithm to replace the pre-configured algorithm, which achieves the aim of game-based learning.
The third component are maze environment, it contains multiple types and cases. In this project, the mazes are made of actual planks. The robot configured with the maze solving algorithm will be tested in these maze environments.

\subsection{Project Plan}
The schedule of this project is based on the aim of project, which has been described in project description. The Gantt chat is shown as Fig. 1.1. In the whole process of this project, only one member participate in, which is the author of this report.
The first milestone is to design an algorithm which can guide the robot to out of maze. The first stage takes five weeks and the designed maze solving algorithm is stated in chapter 2 in detail. 
The second milestone is to implement the basic actions of robot. The robot actions contain basic movement, making a turn, and wall detection. At this stage, the major work relates to programming with the API of robot, the actions programming modules will be output when the second milestone is achieved.
The third milestone contains two tasks, algorithm implementation and configuration in robot. In this stage, the designed algorithm in milestone 1 tends to be implemented in python. Once it has been implemented, the related code will be configured in actual robot for later stage of tests. 
Then, to achieve the milestone of test, the robot configured with related algorithm code will be tested in actual maze environments. Tests are intended to carry out in different mazes for multiple times. The success rate of goal achieving will be counted.
The last milestone are feedback collection and documentation. At this stage, multiple users try out the delivered product and feedbacks will be collected. In the end, the documentation of this project is supposed to be completed.

\figpdf{Gantt chart of project}{fig1}{14cm}

\subsection{Risk Assessment}
The major possible risks to meet and aversion measures are shown in Table 1.1.

\begin{table}[h]
\label{tab:tab1}
\centering
\caption{ Title Of Table }
\renewcommand{\arraystretch}{2}
\setlength{\tabcolsep}{10pt}
\begin{tabular}{ | m{2cm} | m{3cm} | c | c | m{3cm} |} 
\hline Risk & Description & Probability & Severity & Actions to Minimize Risk \\ 
\hline Robot break down & The hardware of robot breaks down in the process of project development. & Moderate & Catastrophic & During the development of project, developer takes care of project equipment. Besides, developer can prepare a spare robot if development funds are adequate. \\ 
\hline Schedule risk & Due to assignments in other courses, the project fails to run as schedule expected. & Likely & Major & Try to finish ahead of time, and balance well with other courses. \\ 
\hline Technologies applied failure & The knowledge of controlling robot needs large amount of learning, developer fails to learn in time. & Likely & Major & Developer specially arranges time for learning new techniques. \\ 
\hline 
\end{tabular} 
\end{table}

\chapter{Design}\label{cpt:des}
\section{Design methodology}
This development of this project is about to adopt incremental model. The product will be designed, implemented and tested incrementally, until the project is completed. The product consists of multiple components, each of which is designed and built separately.
The increments refered to a series of releases, and each increment will provides more functionalities to users \cite{Bell:2016vq}. After the first increment, a version which can be used by the users is delivered. Then, the plan for next increment is developed based on user feedback.
This process will continues until the complete product is delivered.
%这里要插入图片

\section{Data Structure}
The MSA use stack to store the current path of robot. Stack is chosen because its property of First-In Last-Out (FILO). The first element put in the stack will be the last element to be pop. After the robot reaches the destination, the position list poped by the stack is exactly the path from start position to end position. 
Besides, tree structure is also can used in this project for addtional path drawing functionality. While robot is moving in the maze, the position of robot can be recorded in tree structure as nodes. Once the algorithm execute completely, all paths in the maze have been recorded in tree structure.


\section{Algorithm Design}
The designed maze solving algorithm MSA in this project use the recursive models. In each recursive step, it will detect if the destination has been reached. If not, MSA will choose the next available position to move. 
The search for available direction is carried out in the order of left, positive, right and back. If it exists available exploration direction, the robot will move to the next point in that available direction and regard this position as new start position.
In this process, MSA will guide robot explore as deep as it can until the current path of exploration is not available. MSA will explore another path from last crosswise. The whole process is repeated until the robot reaches the destination. 

The pseudocode and flow chart of MSA is shown as Fig. 2.1 and Fig. 2.2:

\figpdf{Pseudocode of MSA}{fig2}{14cm} 

\figpdf{Flow chart of MSA}{fig3}{12cm}

\section{Evaluation Design}

\subsection{Test}
In the test stage, The first task is to perform functional testing from low level to high level. 
First, the each unit in this project is intended to be tested. Developer will first validate the correctness of each unit and check if the codes are in a canonical format. 
Then, the functionality of each module will be test. For instance, the action module of robot will be test interms of moving, making a turn.
The third stage is intergration test. The functionality of several modules that depend on each other will be tested.
Finally, the function of the entire project is supposed to test. The robot configured with designed maze solving algorithm (MSA) should operate in the actual maze.
The tests will be repeated multiple times in different maze environment. In each type of maze, tester continues to test and record the time and the solution path.
In the end, the succeess rate of experiments will be recorded.

In addition, according to project specification, this project allows users to modify the pre-configured algorithm and design their own algorithm. To test this requirement, some programming enthusiasts are intended to be invited to use this product.
They modify the program code and try other algorithms, then the feedbacks which mentioned in project specification are collected.

To evaluate the project, the first is to evaluate if this project satisfies the project specification.
The aims proposed in the project specification:
\lists{number}{
	\item Design a maze solving algorithm and work accurately in robot.
	\item The delivered product allows users design their own algorithms and implement in robot.
}

\subsection{Design a maze solving algorithm and work accurately in robot}
In the previous test stage, tester carry out a lot of experiments and the results have been recorded. The evaluation of the designed algorithm MSA will be based on its success rate, which is calculated as:
\figpdf{Succeess rate equation}{equ1}{12cm}

\subsection{The delivered product allows users design their own algorithms and implement in robot}
To validate this aim, the feedbacks from the users which have been collected in previous test stage will be considered:
\lists{number}{
\item is it user-friendly to modify the program code?
\item Whether they are satisfied with programmable functionality.
}

\chapter{Review against Plan}
Generally, the process of this project follows the plan Fig.3.1. The work progresses to the point indicated by the red line in the Gantt chart. The first milestone Algorithm Design has been completed and status has been updated. 
Besides, the Implementation of Robot Basic Actions is under way. The remaining work shown in Gantt chart will be completed in following semester.

\figpdf{Flow chart}{fig4}{14cm}


% 参考文献列表,请打开 reference.bib 文件添加 bibtex 格式的参考文献
\bibliographystyle{IEEEtran}
\bibliography{reference}
\addcontentsline{toc}{chapter}{Bibliography}

\end{document}

