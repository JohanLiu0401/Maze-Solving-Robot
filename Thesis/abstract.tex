\cleardoublepage
\newgeometry{left=3.2cm,right=2.3cm,top=1cm,bottom=0.7cm} % Change margins locally
\begin{vcentrepage}
\noindent\rule[2pt]{\textwidth}{0.2pt}\\
%\begin{center}
%{\Large\textbf{Spectroscopic Tools for Quantitative Studies of DNA Structure and Dynamics}}
%\end{center}

{\large\textbf{Resumé:}\\}
As computer and mobile technologies advance, a amount of games for educational purposes have been used among learners of different levels, which is called game-based learning(GBL).
It has become the best solution for soft skills learning when traditional learning is homiletic, expensive and difficult to implement.
In this paper, a game-based learning application is developed for those people who wish to learning programming knowledge. 
The application allows user control a tracked mobile robot to get out of maze with their programming skills. 
The robot is built based on Raspberry Pi 3+ and configured with the Debian operating system.
Running this application, tracked mobile robot can explore the maze and find a route out of the maze. In this application, Depth-First-Search algorithm is utilized, which is the core idea of algorithm to exit the maze.
Furthermore, This application is modularized, which also supports users to modify the application for secondary development.

\noindent\rule[2pt]{\textwidth}{0.8pt}
\end{vcentrepage}

\input{abstract_danish}

\restoregeometry
