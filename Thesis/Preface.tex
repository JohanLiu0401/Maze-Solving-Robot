\thispagestyle{empty}
%\hfill
\vspace*{3cm}
\noindent\Huge\textsc{Preface}\\
\normalsize
\noindent\rule[2pt]{\textwidth}{0.8pt}
\hspace*{3cm}

 This thesis was submitted to the Faculty of Science, University of Copenhagen, as a partial fulfillment of the requirements to obtain the PhD degree. The work presented was carried out in the years 2009-2012 partially in the laboratory of Prof. Kristine Kilså at the Department of Chemistry University of Copenhagen and partially at the Department of Chemical and Biological Engineering at Chalmers in the laboratories of Assoc. Prof. Marcus Wilhelmsson and Prof. Bo Albinsson. I've always enjoyed my time at Chalmers where I spent about six months combined during the PhD. I additionally spent three extremely experiencing months in the biochemical laboratory of Prof. Daniel Herschlag at Stanford University, CA, during the spring of 2010. Since no papers resulted from the collaboration with the Herschlag lab the work carried out while I was at Stanford and during the following months is not included in the thesis.

 The thesis is a direct continuation of my Master's thesis titled \emph{"Characterization and Use of Fluorescent Nucleobase Analogues"} which was carried out in part at Chalmers and defended in 2009. The paper resulting from my Master's work, published in JACS in 2009, is highly relevant and frequently referred to in this thesis and it has therefore been included as Appendix 1.\\

\Large\textsc{Thesis objectives}\normalsize

 The objectives of the work reported in this thesis are primarily to 1) design new fluorescent probes based on synthetic DNA base analogues, 2) characterize their photophysical properties in DNA and 3) demonstrate how such probes can be exploited for studying the structure and dynamics of nucleic acids. In addition, a continuing goal is to improve and expand the quantitative Förster resonance energy transfer (FRET) toolbox in a more general sense, which includes the development of generic methodologies and software packages for the simulation and analysis of quantitative time-resolved fluorescence and FRET experiments.\\

\newpage
\Large\textsc{Thesis outline}\normalsize

 This thesis is in the form of a synopsis with attached published papers. While Part I of the thesis provides background information and an overview of the attached papers Part II constitutes the papers.

 In Part I, the first chapter introduces the rapidly progressing research field of fluorescence-based biophysical technologies and explains why this field is important particularly in the life sciences. The next chapter provides an introductory level foundation to the theoretical, experimental and computational concepts used throughout the thesis. Finally, a short explanatory summary of each published paper and software is provided in chapter \ref{chap:PaperOverview}. Part II presents the papers in the order of theme: First papers regarding FRET are presented (Paper I, II, III, and IV), followed by the characterization of existing DNA base analogues (Paper V, as well as paper II and VI), finalizing with the development of new fluorescent base analogues (Paper VI).
%\vspace*{3cm}
%\begin{center}
%%\rule{5cm}{0.2mm}\\
%Søren Preus
%\end{center}
